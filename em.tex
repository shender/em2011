\documentclass[12pt]{article}
% math symbols
\usepackage{amssymb,amsmath}
% for different compilers
\usepackage{ifpdf}
% geometry of page
\usepackage[margin=2.1cm]{geometry}
% float pictures
\usepackage{wrapfig}

% if pdflatex, then
\ifpdf
 \usepackage[english,russian]{babel}
 \usepackage[utf8]{inputenc}
 \usepackage[unicode]{hyperref}
 \usepackage[pdftex]{graphicx}
% if xelatex, then
\else
% math fonts
 \usepackage{fouriernc}
% xelatex specific packages
 \usepackage[xetex]{hyperref}
 \usepackage{xunicode}	% some extra unicode support
 \usepackage{xltxtra}	% \XeLaTeX macro
 \defaultfontfeatures{Mapping=tex-text}
 \usepackage{polyglossia}	% instead of babel in xelatex
 \setdefaultlanguage{russian}
% fonts
 \setromanfont{Charis SIL}
 \setsansfont{OfficinaSansC} 
 \setmonofont{Consolas}
\fi

% several pictures in one figure
\usepackage{subfig}
% calc in TeX expressions
\usepackage{calc}
% nice pictures and plots
\usepackage{pgfplots,tikz,circuitikz}
% different libraries for pictures
\usetikzlibrary{%
  arrows,%
  calc,%
  patterns,%
  decorations.pathreplacing,%
  decorations.pathmorphing,%
  decorations.markings%
}
\tikzset{>=latex}

% colors of the hyperlinks
\hypersetup{colorlinks,%
  citecolor=blue,%
  urlcolor=blue,%
  linkcolor=red
}

\tolerance=1000
\emergencystretch=0.74cm

\newcommand{\nn}{\nonumber}
\newcommand{\pt}{\partial}
\newcommand{\eps}{\epsilon}
\newcommand{\vareps}{\varepsilon}
\newcommand{\const}{\mathrm{const}}
\newcommand{\com}[1]{{\Large{\texttt{{\color{red}(#1)}}}}}

\newcommand{\grad}{\mathrm{grad}\,}
\newcommand{\rot}{\mathrm{rot}\,}
\renewcommand{\div}{\mathrm{div}\,}
\newcommand{\vn}{\vec{\nabla}}


\begin{document}

\section{Уравнение Максвелла и преобразования Галилея.}
% \label{sec:maxwell_galileo}

% \subsection{Галилеевски–инвариантные уравнения Максвелла.}
% \label{sec:maxwell_inv_gal}

% Рассмотрим такую систему уравнений: 

% \begin{eqnarray}
%   \label{eq:maxwell_cut}
%   \div \vec{E} &=& \rho,\\
%   \rot \vec{E} &=& 0,\\
%   \div \vec{B} &=& 0,\\
%   \rot \vec{B} &=& \vec{j} + \frac{\pt \vec{E}}{\pt t}.
% \end{eqnarray}

% Видно, что от обычных уравнений Максвелла они отличаются только
% отсутствием закона Фарадея (в наших обозначениях это должно быть
% слагаемое $-1/c^2 \pt \vec{B} / \pt t$ в правой части второго
% уравнения). 

% Из этих уравнений можно увидеть закон сохранения заряда, применив
% дивергенцию к последнему уравнению. Получится: 

% \begin{equation}
%   \label{eq:conserv_charge}
%   \frac{\pt \rho}{\pt t} + \div \vec{j} =0.
% \end{equation}

% То есть, это самосогласованные уравнения, даже при том, что источники
% зависят от времени. Кроме того, это уравнения, инвариантные
% относительно преобразований Галилея. Действительно, сделаем
% преобразование

% \begin{equation}
%   \label{eq:galileo_transf}
%   \vec{r}' = \vec{r} + \vec{v} t, \quad t'=t.
% \end{equation}

% Что произойдёт при этом с операторами? Вот что: 

% \begin{equation}
%   \label{eq:operators_transf}
%   \vn ' = \vn, \quad \frac{\pt}{\pt t'} = \frac{\pt}{\pt t} -
%   \vec{v}\cdot \vn.
% \end{equation}

% Плотность заряда останется неизменной (это же скаляр), а вот вектор
% тока преобразуется: 

% \begin{equation}
%   \label{eq:sources_transf}
%   \rho' (\vec{r'},t') = \rho (\vec{r},t), \quad \vec{j}' (\vec{r}',t')
%   = \vec{j} (\vec{r},t) + \rho(\vec{r},t) \vec{v}.
% \end{equation}

% Проверим, что если поля преобразуются вот так

% \begin{equation}
%   \label{eq:fields_transf}
%   \vec{E}' = \vec{E}, \quad \vec{B}' = \vec{B} + \vec{v} \times \vec{E},
% \end{equation}
% то уравнения Максвелла инвариантны. Заметим попутно, что последнее
% соотношение означает, что даже если в начальной СО магнитное поле
% отсутствовало ($\vec{B}=0$), то в новой СО оно появится. 

% В самом деле, во-первых, заметим, что 
% \begin{equation}
%   \label{eq:two_maxwell}
%   \div \vec{E}' = \rho, \quad \rot \vec{E}' =0. 
% \end{equation}

% Что касается остальных двух уравнений, то нужно вспомнить два таких
% тождества: 

% \begin{eqnarray}
%   \label{eq:two_identities}
% \nn
%   \div (\vec{a} \times \vec{b}) &=& \vec{b} \cdot \rot \vec{a} -
%   \vec{a} \cdot \rot \vec{b},\\
% \rot (\vec{a} \times \vec{b}) &=& \vec{a} \cdot \div \vec{b} - \vec{b}
% \cdot \div \vec{a} + (\vec{b} \cdot \vn)\vec{a} - (\vec{a}\cdot \vn)\vec{b}.
% \end{eqnarray}

% С их помощью получается, что

% \begin{equation}
%   \div \vec{B}'	= \div(\vec{B}+\vec{v}\times \vec{E}) = \div \vec{B} -
%   \vec{v} \cdot \rot \vec{E} = \div \vec{B}.
% \end{equation}

% А также

% \begin{equation}
%   \rot \vec{B}' = \rot (\vec{B} + \vec{v} \times \vec{E}) = \rot
%   \vec{B} + \vec{v} \cdot \div \vec{E} - (\vec{v}\cdot\vn)\vec{E} =
%   \rot \vec{B} +  \rho \vec{v}  - (\vec{v} \cdot \vn) \vec{E}.
% \end{equation}

% Вспомним теперь соотношение для вектора тока $\vec{j}$
% \eqref{eq:sources_transf}. Видно, что с их помощью, а также с помощью
% \eqref{eq:operators_transf}, последнее уравнение можно переписать как

% \begin{equation}
%   \rot \vec{B}'	= \vec{j}' + \frac{\pt \vec{E}'}{\pt t'}.
% \end{equation}

% Таким образом, мы получили, что все «уравнения Максвелла» остаются
% инвариантными при преобразованиях Галилея. То есть, закон индукции
% Фарадея — чисто релятивистский эффект. 

% Отметим, кстати, забавное свойство, которое следует из всего
% предыдущего: 

% \begin{equation}
%   \label{eq:pt_B}
%   \frac{\pt \vec{B}}{\pt t} = \vn \times (\vec{v} \times \vec{B}).
% \end{equation}


\subsection{Два тождества векторного анализа.}
\label{sec:vector_identities}

\begin{eqnarray}
  \label{eq:two_identities_1}
  \vn \cdot (\vec{a} \times \vec{b}) &=& \vec{b} \cdot ( \vn \times \vec{a}) -
  \vec{a} \cdot (\vn \times \vec{b}),\\
  \label{eq:two_identities_2}
\vn \times (\vec{a} \times \vec{b}) &=& \vec{a} \cdot (\vn \cdot \vec{b}) - \vec{b}
\cdot (\vn \cdot \vec{a}) + (\vec{b} \cdot \vn)\vec{a} - (\vec{a}\cdot \vn)\vec{b}.
\end{eqnarray}


\subsection{Уравнения Максвелла из электростатики.}
\label{sec:maxwell_statics}

Начнём со статических уравнений Максвелла. 

\begin{eqnarray}
  \label{eq:maxwell_statics}
  \vn \cdot \vec{E} &=& \rho, \quad \frac{\pt \rho}{\pt t}=0,\\
\vn \times \vec{E} &=& 0, \quad \frac{\pt \vec{E}}{\pt t}=0.
\end{eqnarray}

Попробуем сделать их зависимыми от времени. Пусть все источники
двигаются равномерно со скоростью $\vec{v}$. Есть альтернативная СО, в
которой источники покоятся. Переход в эту СО осуществляется так: 

\begin{equation}
  \label{eq:transf_galileo}
  \frac{\pt}{\pt t} \to \frac{\pt}{\pt t'} = \frac{\pt}{\pt t} +
  \left( \vec{v} \cdot \vn \right).
\end{equation}

Например, посмотрим, что происходит с зарядом. 

\begin{equation}
  \label{eq:transf_charge}
  0 = \frac{\pt \rho}{\pt t} \to \frac{\pt \rho}{\pt t'} = \frac{\pt
    \rho}{\pt t} + \left( \vec{v} \cdot \vn \right) \rho.
\end{equation}

Так как скорость постоянна, то это можно упростить: 

\begin{equation}
  \label{eq:charge_consrv}
  \frac{\pt \rho}{\pt t} + \vn \left( \rho \vec{v} \right) =0 .
\end{equation}

Это — закон сохранения заряда (при этом $\vec{j} \equiv \rho \vec{v}$
— вектор тока). 

Аналогичное преобразование можно проделать с условием статичности
$\vec{E}$:

\begin{equation}
  \label{eq:E_field_transf}
  0 = \frac{\pt \vec{E}}{\pt t} \to \frac{\pt \vec{E}}{\pt t'} =
  \frac{\pt \vec{E}}{\pt t} + \left( \vec{v} \cdot \vn \right) \vec{E}.
\end{equation}

Теперь используем тождество \eqref{eq:two_identities_2} для упрощения
втрого слагаемого. При этом надо помнить, что $\vec{v}$ постоянна.

\begin{equation}
  0 = \frac{\pt \vec{E}}{\pt t} + \vec{v} \cdot \div \vec{E} - \vn
  \times \left( \vec{v} \times \vec{E} \right) = \frac{\pt
    \vec{E}}{\pt t} + \vec{j} - \vn 
  \times \left( \vec{v} \times \vec{E} \right).
\end{equation}

Можно ввести обозначение $\vec{B} \equiv \vec{v} \times \vec{E}$,
тогда это уравнение приобретёт просто вид 4-го уравнения Максвелла: 

\begin{equation}
  \label{eq:maxwell_4}
  \vn \times \vec{B} = \vec{j} + \frac{\pt \vec{E}}{\pt t}.
\end{equation}

Коль скоро у нас появился новый вектор, надо всё про него
разузнать. Например, какая у него дивергения? 

\begin{equation}
  \label{eq:B_field_div}
  \vn \cdot \vec{B} = \vn \times \left( \vec{v} \times \vec{E} \right)
  = - \vec{v} \cdot \left( \vn \times \vec{E} \right)=0.
\end{equation}

Более того, можно узнать зависимость от времени. Тут надо вспомнить,
что в движущейся СО заряды и поля не зависят от времени, так что

\begin{equation}
  \label{eq:B_field_pt_t}
  0=\frac{\pt \vec{B}}{\pt t'} = \frac{\pt \vec{B}}{\pt t} + \left( \vec{v}
    \cdot \vn \right) \vec{B}.
\end{equation}
Или, опять используя \eqref{eq:two_identities_2}, получаем
\begin{equation}
  \frac{\pt \vec{B}}{\pt t} = \vn \times \left( \vec{v} \times \vec{B} \right).
\end{equation}

Итак, что мы видим? Если заряды двигаются, то это приводит к факту
появления дополнительного поля $\vec{B}$, которое выражается через ток
и через производную электрического поля. Всё остальное почти не
меняется (разве что у плотности повляется зависимость от времени). 

А где же закон индукции Фарадея? Почему-то он не появился в нашем
вычислении. Это связано с тем, что уравнения Максвелла без закона
Фарадея инвариантны относительно преобразований Галилея, так что для
его появления нужно привлечь экспериментальный факт — а именно,
существование электромагнитных волн. Этот факт очевидным образом
нарушает галилеевскую инвариантность. 

\subsection{Электромагнитная волна и закон индукции.}
\label{em_wave}

Потребуем, чтобы существовали электромагнитные волны. Такие волны
обладают свойством

\begin{equation}
  \vec{E} \sim f (x-ct).
\end{equation}

То есть, для них должно выполняться уравнение вида

\begin{equation}
  \label{eq:wave_equation}
  \vn^2 \vec{E}  = \frac{1}{c^2}
  \frac{\pt^2\vec{E}}{\pt t^2}.
\end{equation}

Разумеется, мы требуем, чтобы это уравнение выполнялось вдалеке от
источников ($\rho=0$), то есть, в таких областях, что $\div \vec{E} = 0$. В этих
областях можно преобразовать левую часть: 

\begin{equation}
  \vn^2 \vec{E} = - \vn \times (\vn \times \vec{E}).
\end{equation}

Правая же часть тоже преобразуется (помним о том, что $\vec{j}=0$, а
также используем \eqref{eq:B_field_pt_t}):

\begin{equation}
  \frac{1}{c^2}  \frac{\pt^2\vec{E}}{\pt t^2} = \frac{1}{c^2}
  \frac{\pt}{\pt t} \vn \times \vec{B} = \frac{1}{c^2} \vn \times \left[
    \vn \times \left( \vec{v} \times \vec{B} \right) \right].
\end{equation}

Таким образом, волновое уравнение \eqref{eq:wave_equation} выполняется
в том случае, если

\begin{equation}
  \vn \times \vec{E} = -\vn \times \left( \frac{\vec{v}}{c^2} \times
    \vec{B} \right) = -\frac{1}{c^2} \frac{\pt \vec{B}}{\pt t}. 
\end{equation}

Это — как раз нужный закон Фарадея. 

В итоге, всё выглядит примерно так: 
\begin{enumerate}
\item Начинаем с электростатики;
\item Переходим к движущимся электрическим зарядам;
\item Видим, что для самосогласованности нужно новое поле $\vec{B}$;
\item Получается 4-е уравнение Максвелла;
\item Требуем, чтобы были электромагнитные волны;
\item Получается 2-е уравнение Максвелла (закон Фарадея).
\end{enumerate}


\end{document}
