\documentclass[a4paper,12pt]{article}
% math symbols
\usepackage{amssymb,amsmath}
% for different compilers
\usepackage{ifpdf}
% geometry of page
\usepackage[margin=2.1cm]{geometry}
% float pictures
\usepackage{wrapfig}

% if pdflatex, then
\ifpdf
 \usepackage[russian]{babel}
 \usepackage[utf8]{inputenc}
 \usepackage[unicode]{hyperref}
 \usepackage[pdftex]{graphicx}
% if xelatex, then
\else
% math fonts
 \usepackage{fouriernc}
% xelatex specific packages
 \usepackage[xetex]{hyperref}
 \usepackage{xunicode}	% some extra unicode support
 \usepackage{xltxtra}	% \XeLaTeX macro
 \defaultfontfeatures{Mapping=tex-text}
 \usepackage{polyglossia}	% instead of babel in xelatex
 \usepackage{indentfirst}	% 
 \setdefaultlanguage{russian}
% fonts
 \setromanfont{Charis SIL}
 \setsansfont{OfficinaSansC} 
 \setmonofont{Consolas}
\fi

% several pictures in one figure
\usepackage{subfig}
% calc in TeX expressions
\usepackage{calc}
% nice pictures and plots
\usepackage{pgfplots,tikz,circuitikz}
% different libraries for pictures
\usetikzlibrary{%
  arrows,%
  calc,%
  patterns,%
  decorations.pathreplacing,%
  decorations.pathmorphing,%
  decorations.markings%
}
\tikzset{>=latex,%
  marrow/.style={postaction={draw,decorate,decoration={markings,
    mark=at position 0.6 with {\arrow{latex}}}}}}

% colors of the hyperlinks
\hypersetup{colorlinks,%
  citecolor=blue,%
  urlcolor=blue,%
  linkcolor=red
}

\tolerance=1000
\emergencystretch=0.74cm

\numberwithin{equation}{section}


\newcommand{\nn}{\nonumber}
\newcommand{\pt}{\partial}
\newcommand{\eps}{\epsilon}
\newcommand{\vareps}{\varepsilon}
\newcommand{\const}{\mathrm{const}}
\newcommand{\com}[1]{{\Large{\texttt{{\color{red}(#1)}}}}}

\newcommand{\grad}{\mathrm{grad}\,}
\newcommand{\rot}{\mathrm{rot}\,}
\renewcommand{\div}{\mathrm{div}\,}
\newcommand{\vn}{\vec{\nabla}}


\begin{document}

\section{Введение.}
\label{sec:intro}

Мы собираемся изучать электродинамику --- науку о движении зарядов,
токов и электромагнитных полей. Прежде чем углубляться в теорию,
поймём, какие именно экспериментальные, наблюдаемые явления мы хотим
объяснить. 

\subsection{Экспериментальные факты.}
\label{sec:exp_facts}

Чтобы понимать, к чему мы стремимся, опишем ряд экспериментальных
явлений, которые мы хотим объяснить. 

\begin{enumerate}
\item Отклонение стрелки компаса при прохождении по лежащему рядом
  проводнику электрического тока. Это явление было замечено в 1820
  году датским физиком Эрстедом. Стрелка компаса отклоняется тем
  сильнее, чем больший ток проходит по проводу.
\item Притяжение двух электрических проводов, по которым идёт в одну
  сторону электрический ток. Отталкивание этих проводов, если токи
  текут в противоположные стороны.
\item Существование электромагнитных волн --- то есть, например,
  света. Как распространяется свет? С какой скоростью? Можно ли
  предсказать, какова будет скорость света в данной среде?
\item Возникновение переменной ЭДС в замкнутом проводнике, когда сквозь него
  пролетает магнит.
\end{enumerate}

\subsection{Стратегия.}
\label{sec:strategy}

Как мы будем действовать? Сходу кажется, что в нашем распоряжении нет
никаких инструментов для объяснения этих фактов. Действительно, что мы
знаем об электричестве в целом? Мы знаем, что есть электростатика, то
есть, наука о взаимодействии \textit{статических} зарядов. В рамках
этой науки мы можем установить такие законы, как закон Кулона (о силе
притяжения между двумя точечными зарядами), закон Гаусса (о потоке
электрического поля в зависимости от электрического заряда) и ряд
других. Могут ли они нам помочь в задаче описания движущихся зарядов и
движущихся полей? 

Ключевой особенностью здесь является концепция \textbf{поля}. На
примере электростатики мы видим, что поле появляется уже в простейшей
задаче о взаимодействии двух точечных зарядов. Действительно,
известно, что точечный заряд $q$ создаёт вокруг себя поле,
напряжённость которого выражается формулой 

\begin{equation}
  \label{eq:q_E}
  \vec{E} (\vec{r}) = k \frac{q \vec{r}}{r^3}.
\end{equation}

Видно, что вектор $\vec{E}$ существует в любой точке пространства, вне
зависимости от того, насколько далеко мы отошли от заряда. Это
позволяет ввести понятие \textit{векторного поля} --- вида материи,
который существует при наличии источника. В данном случае источником
электрического поля является заряд $q$. Позднее мы увидим, что
магнитное поле, несмотря на отсутсвие одиночных источников, также
допускает такую интерпретацию. С этого момента мы будем говорить не о
напряжённости $\vec{E}$, а об электрическом (или магнитном) поле
$\vec{E}(x,y,z,t)$ (или $\vec{B}(x,y,z,t)$). Заметим, что в нашем
описании поле может зависеть как от точки пространства, так и от
времени.

Поскольку вектор определяется своими проекциями, то задать векторное
поле --- то же самое, что задать три его проекции. Таким образом,
электрическое поле $\vec{E}$ --- три функции четырёх переменных. 

Таким образом, для того, чтобы научиться описывать электромагнитные
явления, нужно будет научиться работать с полями. Этому будет посвящён
раздел \ref{sec:vector_analysis}. Далее, в разделе
\ref{sec:electrostatics} мы вспомним электростатику и перепишем её
основные формулы на языке векторного анализа. После этого в разделе
\ref{sec:maxwell} мы попробуем перейти к движущимся зарядам путём
перехода в движущуюся инерциальную систему отсчёта.

В разделе \ref{sec:magnetostatics} мы посмотрим на получившуюся в
итоге \textbf{магнитостатику} --- науку о магнитных полях в
статическом приближении. Далее, мы встроим в нашу картину мира
электромагнитные волны (в разделе \ref{sec:em_waves}) и построим
полную систему уравнений Максвелла. 

\section{Векторный анализ.}
\label{sec:vector_analysis}

\section{Электростатика: краткое напоминание.}
\label{sec:electrostatics}

\section{Уравнения Максвелла.}
\label{sec:maxwell}

\subsection{Производная рыбака.}
\label{sec:material_derivative}

Представим себе такую ситуацию: рыбак сидит на берегу реки и изучает
какие-то параметры речной воды, например температуру какого-то
конкретного элемента объёма. Очевидно, эта температура $T$ будет
зависеть от времени $t$ (например, из-за нагрева воды солнцем), но
также и от местоположения $\vec{r}$, так как объём воды перемещается
со скоростью течения $\vec{v}$. В итоге чтобы проследить
\textit{полное} изменение этой температуры во времени нужно следить за
её \textit{полной} производной: 

\begin{equation}
  \label{eq:def_material_derivative}
  \frac{d T(\vec{r},t)}{dt} = \frac{\pt T (\vec{r},t)}{\pt t} +
  \vec{v} \cdot \vn T (\vec{r},t).
\end{equation}

Эта полная производная и называется \textbf{производной рыбака} — она
отслеживает полное изменение некоторого поля, которое проходит мимо
наблюдателя со скоростью $\vec{v}$. 

\subsection{Уравнение неразрывности.}
\label{sec:cont_eq}

Теперь представим себе, что мы следим не за температурой объёма воды,
а за плотностью электрического заряда $\rho$. При этом заряд у нас
статический, а вот наблюдатель двигается со скоростью $-\vec{v}$ (то
есть, мы перешли в инерциальную систему отсчёта, которая двигалась
относительно той, в которой заряды покоятся). Очевидно, эта ситуация
физически ничем не отличается от покоящейся — с той лишь разницей, что
теперь если мы захотим проследить за изменением заряда, то нам нужно
будет пользоваться производной рыбака. 

Допустим теперь, что в нашей «рыбачьей» системе отсчёта мы хотим
потребовать, чтобы всё было статично — то есть, чтобы производная
рыбака от плотности зарядов была равна нулю. С такой ситуацией мы
знаем как обращаться, поэтому и проще всего начать именно с неё. Это
приводит к следующему уравнению на плотностью заряда: 

\begin{equation}
  \label{eq:cont_charge_1}
  \frac{\pt \rho}{\pt t} + \vn ( \rho \vec{v}) =0.
\end{equation}

Здесь мы учли, что наша скорость $\vec{v}$ постоянна. Задумаемся над
физическим смыслом второго слагаемого. Ясно, что это — количество
заряда, который проносится через единичную площадку за единичное
время. 

Действительно, пускай вектор $\vec{j}$ определяет количество зарядов,
которое прошло за единичную площадку в единичное время. Этот вектор
направлен вдоль движения зарядов. Если взять маленькую площадку $dS$ в
данном месте провода, то количество зарядов, которое протекло через
неё в единицу времени, равно

\begin{equation}
  \label{eq:def_j_1}
  \vec{j} \cdot \vec{n}\, dS,
\end{equation}
где $\vec{n}$---единичный вектор нормали к $dS$. 

Предположим теперь, что все наши заряды двигаются со средней скоростью
$\vec{v}$. Тогда заряд, прошедший за время $dt$ через площадку $dS$
равен заряду, который содержится в параллелепипеде с основанием $dS$ и
высотой $v\, dt$. Пусть плотность зарядов равна $\rho$, тогда общее
количество этого заряда равно

\begin{equation}
  \label{eq:def_j_2}
  dQ = \rho \, \vec{v} \cdot \vec{n} \, dS\, dt.
\end{equation}

Отсюда видно, что наш вектор $\vec{j}$ может быть записан в виде
$\vec{j} = \rho \vec{v}$. Этот вектор называется \textbf{плотностью
  тока}. С использованием этого вектора наше уравнение
\eqref{eq:cont_charge_1} может быть переписано в таком виде: 

\begin{equation}
  \label{eq:cont_charge_2}
  \frac{\pt \rho (\vec{r},t)}{\pt t} + \vn \cdot \vec{j} \,(\vec{r},t)=0.
\end{equation}

Это — так называемое \textbf{уравнение неразрывности}. Какой у него
физический смысл? 

Рассмотрим какой-нибудь объём $V$, окружённый поверхностью $S$, из
которого утекает электрический заряд $Q$. Как мы только что видели,
измерить утекание этого заряда можно с помощью вектора плотности тока;
утекший заряд в единицу времени равен 

\begin{equation}
  \label{eq:phys_j_1}
  \int \vec{j} \cdot \vec{n}\, dS. 
\end{equation}

Но заряд нигде не может накапливаться; следовательно, утечка заряда,
которую мы померяли с помощью вектора $\vec{j}$, должна равняться
утечке, померянной с помощью измерения заряда $Q$ в два разных момента
времени. Такое измерение отвечает взятию производной $pQ/pt$. То есть,
можно написать

\begin{equation}
  \label{eq:phys_j_2}
  \int \vec{j} \cdot \vec{n}\, dS = -\frac{\pt Q}{\pt t}.  
\end{equation}

Теперь, чтобы привести уравнение \eqref{eq:phys_j_2} в соответствие с
\eqref{eq:cont_charge_2}, вспомним теорему Гаусса--Остроградского: 

\begin{equation}
  \label{eq:phys_j_3}
   \int \vec{j} \cdot \vec{n}\, dS = \int \vn \cdot \vec{j}\, dV.
\end{equation}

Кроме того, заряд $Q$ можно выразить через плотность $\rho$ и объём: 

\begin{equation}
  \label{eq:phys_j_4}
  Q = \int \rho \, dV.
\end{equation}

Теперь мы видим, что наше уравнение записывается в виде 

\begin{equation}
  \label{eq:phys_j_5}
  \int \left( \frac{\pt \rho}{\pt t} + \vn \cdot \vec{j}  \right)\, dV
  = 0.
\end{equation}

То есть, совпадает с \eqref{eq:cont_charge_2}. Таким образом, мы
установили, что уравнение \eqref{eq:cont_charge_2} является,
фактически, законом сохранения заряда. Действительно, оно связывает
вектор плотности тока, который проходит через поверхность, с
изменением заряда внутри объёма, ограниченного этой поверхностью.

\subsection{Рыбак и электрическое поле.}
\label{sec:maxwell_eq_4}

Коль скоро мы разобрались с электрическим зарядом, можно разобраться и
с электрическим полем. Действительно, раз у нас всё статическое ($d\rho
/dt=0$), то и электрическое поле зависеть от времени не
должно. Естественно, измерения этой зависимости должны также проходить
с использованием производной рыбака. То есть, условие выглядит так: 

\begin{equation}
  \label{eq:mat_der_E}
  \frac{d\vec{E}}{dt} = \frac{\pt \vec{E}}{\pt t} + (\vec{v}\cdot \vn)
  \vec{E} = 0.
\end{equation}

Здесь придётся вспомнить тождество «бац-минус-цаб»: 

\begin{equation}
  \label{eq:bac_cab}
  \vn \times \left( \vec{v} \times \vec{E} \right) = \vec{v} \cdot (\vn
  \cdot \vec{E}) - (\vec{v}\cdot \vn) \vec{E} = \frac{\vec{j}}{\eps_0}  -
  (\vec{v}\cdot \vn) \vec{E}.
\end{equation}

Здесь мы использовали закон Гаусса ($\vn \cdot \vec{E} = \rho/\eps_0$). Таким образом, выражая последнее слагаемое в формуле
\eqref{eq:mat_der_E} с помощью формулы \eqref{eq:bac_cab}, получим: 

\begin{equation}
  \label{eq:maxwell_eq_4_1}
  0 = \frac{\pt \vec{E}}{\pt t} + \frac{\vec{j}}{\eps_0} - \vn \times \left(
    \vec{v} \times \vec{E} \right).
\end{equation}

Последнее слагаемое здесь выражает какую-то новую сущность — что-то,
комбинирующее в себе скорость движения зарядов и электрическое
поле. Обозначим эту новую сущность буквой $\vec{B}$: 

\begin{equation}
  \label{eq:def_magnetic}
  c^2\vec{B} \equiv \vec{v} \times \vec{E}. 
\end{equation}

Константа $c$ введена здесь для удобства. В дальнейшем мы придадим ей
ясный физический смысл.

Если мы ввели новую букву, то естественно было бы узнать про неё всё
по максимуму. Вот, например, мы уже умеем описывать её вихрь — он
создаётся вектором плотности тока с одной стороны и меняющимся
электрическим полем с другой. Можно, например, найти дивергенцию этой
буквы:

\begin{equation}
  \label{eq:div_B_1}
  c^2\vn \cdot \vec{B} =  \vn \cdot \left( \vec{v} \times \vec{E} \right).
\end{equation}

У нас образовалось \textit{смешанное произведение} векторов, такого мы
ещё не видели. Попытаемся с ним разобраться. Оператор $\vn$ по смыслу
является производной, а как брать производную от произведения мы
знаем: 

\begin{equation}
  \label{eq:nabla_eq_1}
  \vn \cdot \left( \vec{v} \times \vec{E} \right) = \vn_v \left(
    \vec{v} \times \vec{E} \right)  + \vn_E \left( \vec{v} \times \vec{E} \right).
\end{equation}

Значками $v,E$ подчёркивается, на какой из множителей действует
градиент. Используем тот факт, что множители можно циклически
переставлять: 

\begin{equation}
  \label{eq:nabla_eq_2}
  \vn_v \cdot \left(
    \vec{v} \times \vec{E} \right)  + \vn_E \cdot \left( \vec{v} \times
    \vec{E} \right) = \vec{E} \cdot \left( \vn_v \cdot \vec{v} \right) +
  \vec{v} \cdot \left( \vec{E} \times \vn_E \right) = -\vec{v} \cdot
  \left( \vn \times \vec{E} \right) =0.
\end{equation}

В последних равенствах мы использовали такие факты: 

\begin{itemize}
\item Скорость $\vec{v}$ постоянна, а поэтому $\vn \cdot \vec{v} =0$;
\item В векторном произведении множители можно менять местами с
  заменой знака: $\vec{E} \times \vn = - \vn \times \vec{E}$.
\item Наше электрическое поле изначально было безвихревым, то есть,
  $\vn \times \vec{E}=0$.
\end{itemize}

Таким образом, выяснилось, что дивергенция новой буквы $\vec{B}$ равна
нулю: $\vn \cdot \vec{B}=0$.

Букву $\vec{B}$ предлагается называть для краткости \textbf{магнитным
  полем}; впоследствии мы увидим, почему это название
оправдано. Кстати говоря, для него можно вывести ещё одно интересное
соотношение, которое нам понадобится в дальнейшем. Коль скоро мы
потребовали выполнения равенства $d\vec{E}/dt=0$ (то есть, электрическое
поле должно быть стационарно в системе рыбака), то разумно потребовать
того же для магнитного поля:

\begin{equation}
  \label{eq:db/dt_1}
  0=\frac{d\vec{B}}{dt} = \frac{\pt \vec{B}}{\pt t} + \left( \vec{v}
    \cdot \vn \right) \vec{B}.
\end{equation}

Теперь используем опять равенство \eqref{eq:bac_cab} вместе с тем, что
$\vn \cdot \vec{B}=0$: 

\begin{equation}
  \label{eq:db/dt_2}
  \frac{\pt \vec{B}}{\pt t} = \vn \times \left( \vec{v} \times \vec{B}  \right).
\end{equation}

Пока это равенство имеет чисто математический смысл. Физическое
значение его мы увидим позже. 


\section{Магнитостатика.}
\label{sec:magnetostatics}

Коль скоро мы ввели магнитное поле, уместно будет посмотреть на то,
как оно ведёт себя в простейшей ситуации — в статической. 

Статикой является такой расклад, при котором поля не зависят от
времени, то есть, в нашем случае электрическое поле $\vec{E}$ от
времени не зависит. При этом уравнение для магнитного поля сильно
упрощается: 

\begin{equation}
  \label{eq:magnetostatics}
  \vn \times \vec{B} = \frac{\vec{j}}{c^2\eps_0}.
\end{equation}

Это довольно тонкий момент: как мы увидим, магнитные поля возникают от
токов, а токи --- не что иное, как движущиеся заряды. Следовательно,
статическое магнитное поле --- только приближение. Это приближение
может быть релевантным только в том случае, когда движется большое
число зарядов, которое можно представить как постоянный электрический
поток $\vec{j}$. Таким образом, на самом деле мы изучаем область
постоянных токов, а не постоянных полей. 

Хорошая новость состоит в том, что все эти условия
самосогласованы. Действительно, из уравнения \eqref{eq:magnetostatics}
следует, что ток --- это ротор магнитного поля. Как мы знаем,
дивергенция ротора всегда равна нулю: $\vn \cdot \vec{j} =0$. Из
уравнения неразрывности \eqref{eq:cont_charge_2} следует, что при этом
плотность зарядов не должна зависеть от времени $\pt \rho / \pt t
=0$. Это соответствует тому, что электрическое поле не меняется со
временем, то есть $\pt \vec{E}/ \pt t =0$, то есть, действительно,
выполняется уравнение \eqref{eq:magnetostatics}.

В дальнейшем мы выведем важные следствия из этого уравнения, а пока
попытаемся воспользоваться более простыми соображениями. 

\subsection{Закон Био-Савара-Лапласа.}
\label{sec:biot_savart_law}

Для того, чтобы объяснить первое экспериментальное явление (отклонение
стрелки компаса) нам понадобится вычислить явно магнитное поле от
такого провода. 

\com{Рисунок к вычислению}

Итак, рассмотрим кусок провода длины $dl$, по которому со скоростью
$\vec{v}$ перемещаются заряды, то есть, течёт ток. На расстоянии
$\vec{r}$ от этого кусочка провода мы хотим сосчитать магнитное поле
$d\vec{B}$. Для начала нам нужно найти, какое электрическое поле
создаёт подобный заряд.

Заряд, заключённый в таком участке провода, равен $dq = \rho\, S\,
dl$, где $S$ — площадь сечения куска провода, $\rho$ — плотность
электрического заряда. Таким образом, электрическое поле в точке
наблюдения равно 

\begin{equation}
  \label{eq:biot_savart_1}
  d\vec{E} = \frac{1}{4\pi\eps_0} \frac{dq \cdot \vec{r}}{|\vec{r}|^3}
  = \frac{1}{4\pi\eps_0} \frac{\rho S dl
    \cdot \vec{r}}{|\vec{r}|^3}.
\end{equation}

Теперь умножим скорость движения зарядов $\vec{v}$ векторно на это
электрическое поле: 

\begin{equation}
  \label{eq:biot_savart_2}
  d\vec{B} = \frac{\vec{v}}{c^2}  \times d\vec{E} = \rho \frac{\vec{v}}{4\pi\eps_0c^2} S dl \times
  \frac{\vec{r}}{|\vec{r}|^3} = \frac{I d\vec{l} \times \vec{r}}{4\pi\eps_0c^2|\vec{r}|^3}.
\end{equation}

В последнем равенстве мы использовали тот факт, что $\rho \vec{v} S =
\vec{j} S = \vec{I}$. 

Видно, что эта формула играет в магнитостатике примерно ту же роль,
что в электростатике играет закон Кулона --- по заданному
распределению источников магнитного поля (то есть, токов) даёт
величину магнитного поля. Предлагается по аналогии с законом Кулона
ввести обозначение для коэффициента $1/\eps_0c^2$, который будет нам
ещё неоднократно встречаться. Именно, назовём эту комбинацию
\textbf{магнитной проницаемостью} $\mu_0 = 1/\eps_0c^2$. 


Формула \eqref{eq:biot_savart_2} для магнитного поля, создаваемого
элементом проводника $dl$, по которому течёт ток $I$, называется
\textbf{законом Био-Савара-Лапласа}. Если мы вдруг захотим сосчитать
магнитное поле от всего проводника, нам придётся проинтегрировать эту
формулу по всему проводнику. В некоторых случаях это удаётся сделать,
но в большинстве — увы. Ниже мы рассмотрим несколько примеров удачного
стечения обстоятельств. 

Кстати говоря, заметим, что этот закон объясняет одну из проблем,
сформулированных в самом начале, в разделе \ref{sec:exp_facts}, а
именно, отклонение стрелки компаса рядом с проводом, по которому течёт
ток. Как мы знаем, стрелка компаса реагирует на изменение магнитного
поля; наш закон Био--Савара--Лапласа как раз говорит о том, что когда
по проводнику идёт ток, рядом с ним создаётся магнитное поле. Кроме
того, чем сильнее ток, тем это магнитное поле сильнее. 

\subsubsection{Пример: поле кругового тока.}
\label{sec:ex_current_circle}

Рассмотрим простейший случай — вычисление магнитного поля
от кругового проводника, по которому течёт ток $I$. Магнитное поле мы
будем вычислять в центре этой окружности, используя только что
выведенный закон Био-Савара-Лапласа.

\begin{wrapfigure}{r}{4cm}
\centering
\begin{tikzpicture}
  \draw[thick] (0,0) circle (1.5cm);
  \draw[thick,blue,->] (0,0) -- ++(30:2cm) node[right] {$\vec{r}$};
  \draw[line width=0.15cm] (0,0) ++(40:1.5cm) node[left] {$d\vec{l}$} arc (40:20:1.5cm);
  \draw[thick,->] (-1.5,0) arc (180:140:1.5cm) node[left] {$I$};
  \draw[thick,blue,->] (0,0) -- ++(260:1.5cm) node[right,midway] {$R$};
\end{tikzpicture}
\label{fig:current_circle}
\end{wrapfigure}

Разобьём наш проводник на много мелких кусочков длины $dl$. От каждого
такого кусочка вклад в магнитное поле будет даваться формулой
\eqref{eq:biot_savart_2}. В нашем случае ситуация ещё больше
упрощается: как видно из рисунка, вектора $\vec{r}$ и $d\vec{l}$ всё
время перпендикулярны друг другу, так что их векторное произведение
равно обычному (разумеется, надо ещё принять во внимание направление
получающегося вектора). Таким образом, полное магнитное поле в центре
окружности равно: 

\begin{equation}
  \label{eq:ex_current_circle}
  \vec{B} = \sum d\vec{B} = \frac{\mu_0}{4\pi} \sum \frac{I\, dl\, R}{R^3} =
  \frac{\mu_0 I}{4 \pi R^2} \sum dl = \frac{\mu_0 I}{2R}.
\end{equation}

В последнем переходе мы использовали такой факт: если просуммировать
длины отрезков по всей окружности, то получится полная длина
окружности, то есть, $2\pi R$.

\subsubsection{Пример: поле длинного провода.}
\label{sec:wire_current}

Рассмотрим теперь следующую по сложности ситуацию: магнитное поле от
длинного прямого провода, по которому течёт ток $I$. Опять разобьём
наш провод на много маленьких кусочков. 

\begin{wrapfigure}{r}{4cm}
\centering
\begin{tikzpicture}
  \draw[blue,thick,->] (0,0.75) -- (1.5,1.5) node[below]
  {$\vec{r}$};
  \draw[very thick,->] (0,0) node[below] {$I$} -- (0,2) node[left] {$x$};
  \draw[line width=0.15cm] (0,0.5) -- (0,1) node[midway,left=0.05cm] {$dx$} ;
  \draw[blue] (1.5,1.5) -- (0,1.5) node[above,midway] {$y$};
  \draw[blue,fill=blue] (1.5,1.5) circle (0.05cm) node[above] {$A$};
  \draw[blue] (0,1.1) arc (90:25:0.35cm);
  \draw[blue] (0.3,1.2) node {$\alpha$};
\end{tikzpicture}
\label{fig:current_wire}
\end{wrapfigure}

Угол между кусочком $d\vec{l}$ и радиус--вектором $\vec{r}$ обозначим
за $\alpha$. Видно, что из правила правой руки следует, что в точке
наблюдения (откуда проведён радиус--вектор) магнитное поле будет
направлено за плоскость рисунка. Подобным образом оно будет направлено
для всех кусочков, поэтому в расчёте мы можем забыть про направление
от каждого конкретного кусочка и помнить только то, что итоговое
магнитное поле должно быть направлено от нас. 

Введём координаты: $x$ --- продольная координата вдоль провода,
$y$ --- поперечная. Магнитное поле от кусочка длины $dx$ равно

\begin{equation}
  \label{eq:wire_current_1}
  dB = \frac{\mu_{0}}{4\pi}\,dx \frac{I \sin \alpha
    \sqrt{x^2+y^2}}{\left(x^2+y^2\right)^{3/2}} = \frac{\mu_0}{4\pi} dx \frac{I
  y}{\left(x^2+y^2\right)^{3/2}}.
\end{equation}

Теперь, чтобы вычислить полное поле, мы должны проинтегрировать это
выражение. Так как провод у нас длинный, то интегрировать надо от
$-\infty$ до $+\infty$: 

\begin{equation}
  \label{eq:wire_current_2}
  B(y) = \frac{\mu_0}{4\pi} Iy \int\limits_{-\infty}^{+\infty} \frac{dx}{\left(x^2+y^2\right)^{3/2}} .
\end{equation}

Вычислить этот интеграл сравнительно непросто, однако результат простой: 

\begin{equation}
  \label{eq:wire_current_3}
  B(y) = \frac{\mu_0}{4\pi} \frac{2I}{y}.
\end{equation}

Таким образом, поле длинного провода оказывается убывающим как $1/y$,
где $y$ --- расстояние от точки наблюдения до провода. Вокруг провода
оно закручено по правилу правой руки. 

\subsection{Закон Ампера.}
\label{sec:amperes_law}

В предыдущем примере мы поняли, что даже самые простые на вид задачи
могут порождать сравнительно сложные вычисления. В тоже время, у этого
примера имелась очевидная цилиндрическая симметрия (вдоль оси
провода), но мы ей никак не воспользовались. 

Однако, глядя на уравнение \eqref{eq:magnetostatics}, мы можем сделать
вывод, что воспользоваться этой симметрией вполне
реально. Действительно, перепишем его через циркуляцию: 

\begin{equation}
  \label{eq:amperes_law}
  \int \left( \vn \times \vec{B} \right) \cdot d \vec{S} =
  \mu_0 \int \vec{j} \cdot d\vec{S} \rightarrow  \oint
  \vec{B} \cdot d\vec{l} = \mu_0 I.
\end{equation}

Иными словами, циркуляция магнитного поля по некоторому контуру равна
току, который этот контур пронизывает. Это --- \textbf{закон
  Ампера}. Воспользуемся им для вычисления магнитного поля вокруг
длинного провода с током. 

\begin{wrapfigure}{r}{4cm}
\centering
\begin{tikzpicture}
  \draw[very thick,->] (0,0) -- (0,2) node[left] {$I$};
  \draw[thick,blue] (0,1) ellipse (1cm and 0.5cm);
  \draw[thick,blue,->] (0,0.5cm) arc (270:360:1cm and 0.6cm)
  node[right] {$\vec{B}$};
\end{tikzpicture}
\label{fig:current_wire_field}
\end{wrapfigure}

Возьмём контур в виде окружности радиуса $R$ вокруг провода. По
симметрии, на этом контуре вектор $\vec{B}$ постоянен по модулю, а по
направлению совпадает с касательной к контуру в каждой точке. Таким
образом, циркуляция в законе Ампера \eqref{eq:amperes_law}
превращается в простое произведение:

\begin{equation}
  \label{eq:wire_current_ampere}
  \oint \vec{B} \cdot d\vec{l} = B \oint dl = B\cdot 2\pi R = \mu_0 I.
\end{equation}

Выражая отсюда $B$, получаем полное согласие с ответом, который был
получен намного более сложным образом \eqref{eq:wire_current_3}. Таким
образом, закон Ампера уместно применять тогда, когда в задаче имеется
симметрия, позволяющая сводить вычисление циркуляции к простым
арифметическим действиям. 

\subsubsection{Поле соленоида. }
\label{sec:solenoid}

В качестве ещё одного примера вычисления магнитного поля с помощью
закона Ампера рассмотрим задачу о соленоиде. Пусть имеется длинный
провод, свёрнутый в спираль. Такая спираль и называется
\textbf{соленоидом}. 

Пустим по этому проводу ток $I$. Примем как данность опытный факт:
если соленоид очень длинный, то снаружи его магнитное поле
пренебрежимо мало (вообще это можно доказать, но несколько
муторно). Как найти поле внутри соленоида? 

Из симметрии мы можем ожидать, что внутри соленоида силовые линии
магнитного поля $\vec{B}$ направлены параллельно его оси. 

Возьмём прямоугольный контур $C$ ширины $L$, который охватывает $N$
витков провода. Вычислим циркуляцию поля $\vec{B}$ вдоль этого
контура --- по закону Ампера она связана с током, который пронизывает
этот контур. 

\begin{equation}
  \label{eq:der_mfield_solenoid}
  B \cdot L = \mu_o N I.
\end{equation}

Вводя плотность витков по формуле $n = N/L$, получим для величины
магнитного поля

\begin{equation}
  \label{eq:mfield_solenoid}
  B = \mu_0 n I.
\end{equation}

Согласно одному из уравнений Максвелла, магнитные линии нигде не
начинаются и не заканчиваются. В случае соленоида это эквивалентно
тому, что они выходят из одного его конца и входят в другой
непрерывным образом. У магнитных линий в самом деле нет источника. 

\section{Зачем нужен ток смещения?}
\label{sec:displacement_current}

Со статической ситуацией вроде бы всё понятно. Но остаётся вопрос ---
каков физический смысл добавки $\pt \vec{E} / \pt t$ в уравнении
\eqref{eq:maxwell_eq_4_1}? В статике эта добавка не ловится, и про её
смысл мы ничего сказать не можем. Увидеть её важность можно только в
динамике. 

\begin{figure}[h]
  \centering
  \begin{tikzpicture}
    \draw[very thick,marrow] (0,4) -- (0,2.5) node[right,midway,blue] {$I$};
    \draw[very thick,marrow] (0,1.5) -- (0,0);
    \draw[very thick] (-2,2.5) node[left] {$Q$} -- (2,2.5);
    \draw[very thick] (-2,1.5) node[left] {$-Q$} -- (2,1.5);
    \draw[dashed,blue] (0,3.2) ellipse (2.5cm and 0.5cm);
    \draw[blue] (2.8,3.2) node {$C$};
    \draw[dashed,red] (0,2) ellipse (2.5cm and 0.2cm);
    \draw[red] (2.8,2) node {$C'$};
    \foreach \x in {-0.75,-0.25,0.25,0.75} {\draw[marrow] (\x,2.5) --
      (\x,1.5);};
    \draw[blue,->] (3,0.8) node[blue,below] {$\vec{E}$} to [out=90,in=350] (0.8,1.7);
  \end{tikzpicture}
  \label{fig:displacement}
\end{figure}

Рассмотрим заряжающийся конденсатор, подсоединённый к проводам, по
которым течёт ток $I$. Сначала посмотрим на ситуацию вблизи самих
проводов, вдали от конденсатора. Очевидно, вокруг провода будет
магнитное поле, как мы видели в разделе
\ref{sec:wire_current}. Вычислить его напряжённость можно опять
применяя закон Ампера к контуру $C$. Мы сможем применять теорему
Ампера до тех пор, пока контур не пересечёт пластины
конденсатора. Между пластинами, очевидно, никакого тока нет, и закон
Ампера говорит, что магнитное поле должно равняться нулю. Это было бы
очень странно.

И тут нас как раз выручает добавка, введённая
Максвеллом. Что происходит при зарядке конденсатора? На его обкладках
меняется заряд. Соответственно, между обкладками возникает
\textit{переменное} электрическое поле $\vec{E}(t)$. Это электрическое
поле, согласно уравнению \eqref{eq:maxwell_eq_4_1}, вызывает
циркуляцию магнитного поля. Таким образом, магнитное поле остаётся и
внутри конденсатора. 

Посмотрим на то, как это происходит количественно. Уравнение
\eqref{eq:maxwell_eq_4_1} взятое на контуре $C'$ (который ограничивает
поверхность $S$) даёт

\begin{equation}
  \label{eq:displacement_cur_1}
  \vn \times \vec{B} = \frac{1}{c^2} \frac{\pt \vec{E}}{\pt t}
  \Rightarrow \int \left( \vn \times \vec{B}  \right)  \cdot d \vec{S}
  = \frac{1}{c^2} \frac{\pt}{\pt t} \int \vec{E} \cdot d \vec{S}.
\end{equation}

Левая часть, как мы уже видели, может быть переписана в виде
циркуляции вдоль контура $C'$, а правая часть представляет собой
просто поток электрического поля через поверхность: 

\begin{equation}
  \label{eq:displacement_cur_2}
  \oint \vec{B} \cdot d \vec{l} = \frac{1}{c^2} \frac{\pt \Phi}{\pt t}.
\end{equation}

Как мы знаем из теоремы Гаусса, поток электрического поля равен заряду
на обкладке конденсатора, то есть, $Q(t)/\eps_0$. Пусть наш контур
$C'$ представляет собой окружность радиуса $r$, тогда для циркуляции
вдоль такого контура получаем

\begin{equation}
  \label{eq:displacement_cur_3}
  B \cdot 2 \pi r = \frac{1}{\eps_0c^2} \frac{\pt Q}{\pt t}. 
\end{equation}

В правой части стоит производная заряда по времени, но это и есть ток
$I(t)$! Таким образом, окончательно получается

\begin{equation}
  \label{eq:displacement_cur_4}
  B = \frac{\mu_0 I }{2\pi r},
\end{equation}
что полностью согласуется с выражением для магнитного поля провода,
полученным ранее \eqref{eq:wire_current_ampere}.

\section{Электромагнитные волны  и уравнения Максвелла. }
\label{sec:em_waves}

\subsection{Ещё одно уравнение Максвелла.}
\label{sec:maxwell_eq_4}

Теперь нам нужно в нашу картину мира встроить электромагнитные
волны. Существование этих волн приводит к условию

\begin{equation}
  \label{eq:waves_z_dir}
  \vec{E} \sim f (z-ct).
\end{equation}

В данном случае $z$~---~координата, вдоль которой распространяется эта
волна, $c$~---~скорость волны. Чтобы как-то скомпоновать этот факт с
тем, что мы уже знаем об электрическом и магнитном поле, предлагается
написать на $\vec{E}$ такое дифференциальное уравнение: 

\begin{equation}
  \label{eq:waves_diff_eq}
  \frac{\pt^2}{\pt z^2} \vec{E} = \frac{1}{c^2} \frac{\pt^2}{\pt t^2} \vec{E}.
\end{equation}

Нетрудно видеть, что поле $\vec{E}$ в форме \eqref{eq:waves_z_dir}
действительно удовлетворяет этому уравнению. Отсюда понятно, как
обобщить эту формулу на тот случай, если поле $\vec{E}$
распространяется в произвольном направлении, а не только вдоль оси
$z$: 

\begin{equation}
  \label{eq:waves_arb_dir}
   \left( \frac{\pt^2}{\pt x^2} + \frac{\pt^2}{\pt y^2} + \frac{\pt^2}{\pt
     z^2} \right) \vec{E} \equiv \vn^2 \vec{E} = \frac{1}{c^2} \frac{\pt^2}{\pt t^2} \vec{E}.
\end{equation}

Здесь мы впервые встречаем оператор $\vn^2$. К счастью, его можно
переписать через известные нам объекты, опять используя тождество
<<бац-минус-цаб>>: 

\begin{equation}
  \label{eq:bac_cab_2}
  \vn \times \left( \vn \times \vec{E}  \right) = \vn \cdot \left( \vn
  \cdot \vec{E} \right) - \vn^2 \vec{E}.
\end{equation}

Здесь надо вспомнить, что мы рассматриваем именно электромагнитные
установившиеся волны, то есть, далёкие от источника зарядов. Далеко от
источника нет ни плотности электрического заряда, ни тока. То есть, в
нашем случае $\rho=0, \, \vec{j}=0$. Коли так, то $\vn \cdot \vec{E}
= 4 \pi \rho=0$. Соответственно, наше уравнение
\eqref{eq:waves_arb_dir} превращается в такое: 

\begin{equation}
  \label{eq:waves_eq_2}
  \vn \times \left( \vn \times \vec{E}  \right) = - \frac{1}{c^2} \frac{\pt^2}{\pt t^2} \vec{E}.
\end{equation}

Правая часть также может быть преобразована согласно уравнению
\eqref{eq:maxwell_eq_4_1} (мы используем опять тот факт, что вдали
от источников тока $\vec{j}=0$):

\begin{equation}
  \label{eq:waves_eq_3}
  \frac{1}{c^2} \frac{\pt^2}{\pt t^2} \vec{E} =
  \frac{\pt}{\pt t} \vn \times \vec{B} =  \vn \times \left[
  \vn \times \left( \vec{v} \times \vec{B}  \right)\right].
\end{equation}

Сравнивая уравнения \eqref{eq:waves_eq_2} и \eqref{eq:waves_eq_3},
видим, что для их согласованности надо потребовать

\begin{equation}
  \label{eq:waves_eq_4}
  \vn \times \vec{E} = - \vn \times \left( \vec{v} \times \vec{B}  \right).
\end{equation}

А вот теперь мы можем использовать тождество
\eqref{eq:db/dt_2}. Получится довольно простое соотношение: 

\begin{equation}
  \label{eq:faradays_law}
  \vn \times \vec{E} = -\frac{\pt \vec{B}}{\pt t}.
\end{equation}

Это --- \textbf{закон индукции Фарадея}. Прояснению его физического
смысла будет посвящена отдельная глава. 

Резюмируем все получившиеся результаты. Итак, если заряды находятся в
движении, то верны следующие соотношения для электрического и
магнитного поля: 

\begin{eqnarray}
  \label{eq:maxwell_eqs}
  \vn \cdot \vec{E} &=& \frac{\rho}{\eps_0},\\
  \vn \cdot \vec{B} &=& 0,\\
  \vn \times \vec{E} &=& \phantom{\mu_0 \vec{j}} -\frac{\pt \vec{B}}{\pt t},\\
  \vn \times \vec{B} &=& \mu_0 \vec{j} + \frac{1}{c^2}\frac{\pt
    \vec{E}}{\pt t}.
\end{eqnarray}

Система этих уравнений называется \textbf{уравнениями Максвелла}. Это
основные уравнения электродинамики. Они однозначно описывают динамику
электромагнитной системы. В произвольном случае распределия зарядов и
токов решить их, конечно, сложно. Мы, однако, сосредоточимся на
прояснении физического смысла отдельных слагаемых и эффектов, которые
они описывают. Случаем общего решения этих уравнений мы заниматься не
будем. 

\com{Комментарий про магнитные заряды и соотв. места в уравнениях
  Максвелла}. 

\subsection{Сила Лоренца.}
\label{sec:lorentz_force}

Во время вывода закона индукции мы видели, что между полями $\vec{E}$
и $\vec{B}$ существует ещё одна связь, даваемая уравнением
\eqref{eq:waves_eq_4}. Это уравнение говорит нам, что если в магнитном
поле происходит движение зарядов со скоростью $-\vec{v}$, то
эффективная напряжённость электрического поля равна $\vec{E} =
-\vec{v} \times \vec{B}$. Таким образом, мы можем сказать, что на
такой заряд будет действовать сила

\begin{equation}
  \label{eq:lorentz_force}
  \vec{F} = q \vec{v} \times \vec{B},
\end{equation}
которая называется \textbf{силой Лоренца}. Эта сила действует на все
движущиеся заряды в магнитном поле. 

Заметим, что так как эта сила перпендикулярна скорости (скорость ведь
входит в качестве одного из множителей векторного произведения), то
мощность $P = \vec{F} \cdot \vec{v} $ силы Лоренца равна
нулю. Следовательно, сила Лоренца не совершает никакой работы. 

Итак, если заряженная частица $q$ двигается в электрическом и
магнитном полях с напряжённостями $\vec{E}$ и $\vec{B}$, то полная
сила, которая действует на эту частицу, равна

\begin{equation}
  \label{eq:lorentz_force_full}
  \vec{F} = q \left( \vec{E} + \vec{v} \times \vec{B} \right). 
\end{equation}

\section{Физика индукции.}
\label{sec:induction}

\subsection{Начальные соображения.}
\label{sec:induction_start}



Теперь, когда в нашем распоряжении есть математическая форма закона
индукции Фарадея \eqref{eq:faradays_law} и силы Лоренца
\eqref{eq:lorentz_force}, мы можем обсудить физические эффекты,
соответствующие этим формулам. 

Начнём с силы Лоренца. Возьмём медную проволоку, рядом с которой
вертикально стоит магнит. Концы провода замкнём на гальванометр. Что
будет, если немного подвигать проволоку? 

Мы знаем, что на электроны проводимости в проволоке будет действовать
сила Лоренца. Действитель, есть магнитное поле, обеспечиваемое
наличием магнита, а электроны двигаются с некоторой скоростью за счёт
того, что мы двигаем проволоку. Соответственно, возникнет сила,
направленная перпендикулярно магнитному полю и направлению движения
проволоки. 

Эта сила будет иметь какую-то составляющую вдоль самой проволоки; под
влиянием этой составляющей начнут двигаться электроны проводимости, то
есть, возникнет некий ток, который будет фиксироваться
гальванометром. 

Иными словами, при движении проволоки в магнитном поле за счёт силы
Лоренца возникает ток. 

Предположим теперь, что провод мы не двигаем, а вместо этого двигаем
магнит. В этом случае гальванометр также зафиксирует наличие
эффекта. То есть, опять присутствует некая сила, которая заставляет
двигаться электроны проводимости. Чтобы количественно описать эту
силу, обратимся к закону Фарадея \eqref{eq:faradays_law}. 

Пусть наша проволока $C$ ограничивает некую поверхность
$S$. Проинтегрируем закон Фарадея по этой поверхности: 

\begin{equation}
  \label{eq:eds_1}
  \int_S \left( \vn \times \vec{E} \right) \cdot d\vec{S} = -
  \frac{\pt}{\pt t} \int_S \vec{B} \cdot d\vec{S}.
\end{equation}

Здесь мы воспользовались тем, что поверхность не меняет форму, поэтому
весь интеграл в правой части можно запихать под производную. Теперь
вспомним, что по закону Стокса интеграл от ротора в левой части можно
переписать как циркуляцию: 

\begin{equation}
  \label{eq:eds_2}
  \int_S \left( \vn \times \vec{E} \right) \cdot d\vec{S} = \oint_C
  \vec{E} \cdot d \vec{l}.
\end{equation}

Какой физический смысл у интеграла в правой части? Если бы домножили
его на заряд проводимости $q$, было бы так: 

\begin{equation}
  \label{eq:eds_3}
  \oint_C q \vec{E} \cdot d \vec{l} = \oint_C \vec{F} \cdot d \vec{l},
\end{equation}
то есть, получили бы работу, которую <<сила индукции>> совершает над
зарядами проводимости. Иными словами, это и есть та сила, которая
ответственна за появление тока в проводнике. Сила такого рода,
действующая на единичный заряд, называется \textbf{электродвижущей
  силой}, или, сокращённо, \textbf{ЭДС}. Обычно она обозначается
буквой $\vareps$. 

Перепишем наш закон Фарадея:

\begin{equation}
  \label{eq:eds_4}
  \vareps = - \frac{\pt}{\pt t} \int_S \vec{B} \cdot d\vec{S}.
\end{equation}

Теперь разберёмся с правой частью. Там стоит произведение магнитного
поля на элемент площадки, проинтегрированное по всей
поверхности. Иными словами, это не что иное, как поток магнитного
поля. Итак, закон Фарадея окончательно переписывается в виде

\begin{equation}
  \label{eq:eds_5}
  \vareps = - \frac{\pt \Phi}{\pt t}.
\end{equation}

Он говорит нам о том, что от изменения магнитного потока в проводнике
возникает ЭДС, пропорциональная скорости изменения этого потока. Это
уравнение и объясняет эффект, который впервые наблюдал Фарадей, когда
двигал магнит рядом с электрической цепочкой. 

\subsection{Прямоугольная рамка.}
\label{sec:rectangle}

Теперь разберём пример, который позволит нам досчитать ЭДС до
конца. Рассмотрим проволочную рамку, которая состоит из U--образной
части и подвижной перемычки. 

\begin{figure}[h]
  \centering
  \begin{tikzpicture}
    \draw[very thick] (4,1) -- (0,1) -- (0,-1) -- (4,-1);
    \draw[very thick,fill=gray] (3,-1.1) rectangle (3.1,1.1);
    \draw[blue,thick,->] (3.2,0) -- (3.9,0) node[midway,above]
    {$\vec{v}$};
    \draw[thick,blue,<->] (-0.3,-1) -- (-0.3,1) node[midway,left]
    {$h$};
    \draw[blue] (1.5,0) circle (0.15cm) node[right=0.1cm] {$\vec{B}$};
    \draw[blue,fill=blue] (1.5,0) circle (0.04cm);
    \draw[blue,thick,->] (0.75,-0.75) arc (215:115:1cm) node[right] {$I$};
  \end{tikzpicture}
  \label{fig:rect_b_field}
\end{figure}

Таким образом, электрическая цепь всегда замкнута, но её длина и
площадь может меняться. Поместим эту конструкцию в однородное
электрическое поле так, чтобы плоскость прямоугольника оказалась
перпендикулярна полю. 

Как мы уже видели, при движении перемычки в цепи должна возникать
ЭДС. Действитетельно, в этой перемычке есть какие-то заряды, на
которые будет действовать сила Лоренца. Эта сила равна просто $F = vB$
для единичного заряда (так как магнитное поле и скорость перемычки
перпендикулярны друг другу). Она постоянна вдоль длины перемычки;
таким образом, суммируя её вдоль всей перемычки, получим, что ЭДС,
возникающая в цепи, равна

\begin{equation}
  \label{eq:rect_eds_1}
  \vareps = - F \cdot h = - v B h.
\end{equation}

С другой стороны, если мы представим, что формула \eqref{eq:eds_5}
верна в случае, когда меняется площадь, а магнитное поле остаётся
постоянным, мы получим, что поток магнитного меняется со временем как 

\begin{equation}
  \label{eq:rect_eds_2}
  \Phi = vt\cdot h \cdot B,
\end{equation}
и, таким образом, ЭДС равна 

\begin{equation}
  \label{eq:rect_eds_3}
  \vareps = -\frac{\pt \left( vt \cdot h \cdot B \right)}{\pt t} = - v B h,
\end{equation}
то есть, совпадает с выражением \eqref{eq:rect_eds_1}. Можно доказать,
что это правило потока остаётся верным для любых поверхностей, не
только для прямоугольных. Таким образом, ЭДС равна скорости изменения
потока вне зависимости от того, какова природа этого изменения --- за
счёт изменения магнитного поля или за счёт изменения контура. Нужно,
однако, всегда помнить, что в этом правиле играют одновременно два
фактора: закон индукции Фарадея \eqref{eq:faradays_law} и сила Лоренца
\eqref{eq:lorentz_force}.

Попутно заметим, что сила, действующая на перемычку, оказывается прямо
пропорциональной скорости и направленной в противоположную сторону. То
есть, получается что-то похожее на силу вязкости. Такая сила
получается всякий раз, когда движущиеся проводники создают
индуцированные токи в магнитном поле.

\subsection{Два параллельных провода.}
\label{sec:two_parallel_lines}

Теперь, когда мы знаем, какое действие оказывает магнитное поле на
движущиеся заряды (то есть, на ток), мы можем объяснить ещё один
экспериментальный факт: притяжение (или отталкивание) двух
параллельных проводов с током.

\begin{wrapfigure}{r}{4cm}
\centering
\begin{tikzpicture}
  \coordinate (a) at (2,0); 
  \coordinate (b) at (1,3); 
  \draw[very thick,->] (0,0) -- (-1,3) node[blue,above] {$I_1$};
  \draw[very thick,->] (a) -- (b) node[blue,above] {$I_2$};
  \draw[blue,thick,->] (-1,1) arc (180:45:0.75cm) node[above=0.3cm]
  {$\vec{B}$};
  \coordinate (c) at ($(a)!0.5!(b)$);
  \coordinate (d) at ($(c)!1cm!90:(b)$);
  \draw[thick,blue,->] (c) -- (d) node[below] {$\vec{F}$};
  \coordinate (e) at ($(a)!0.6!(b)$);
  \draw[line width=0.1cm] (c) -- (e) node[blue,right] {$dl$};
\end{tikzpicture}
\label{fig:current_wire_ampere}
\end{wrapfigure}

Действительно, как мы знаем из раздела \ref{sec:wire_current}, длинный
провод с током $I_1$ создаёт вокруг себя поле, закрученное так, как
показано на рисунке. Это магнитное поле будет взаимодействовать с
электронами проводимости во втором проводе, по которому идёт ток
$I_2$. Взаимодействие будет обеспечиваться силой Лоренца
\eqref{eq:lorentz_force}. Поскольку провода параллельны друг другу,
легко понять, что результирующая сила $\vec{F}$ будет перпендикулярна
проводу и лежать в плоскости проводов. Действительно, сила должна быть
перпендикулярна скорости движения электронов и магнитному полю.

Видно, что сонаправленные токи будут притягиваться (сила $\vec{F}$
направлена от второго провода к первому), а разнонаправленные ---
отталкиваться. Тем самым мы объяснили ещё один экспериментальный факт,
заявленный в самом начале.  

\section{Движущееся электромагнитное поле. }
\label{sec:em_plate_moving}

Теперь настало самое время для прояснения физического смысла различных
констант, а также связи всех уравнений Маквселла друг с
другом. Рассмотрим для этого заряженный лист, которая размещается
в плоскости $yz$. Пусть он быстро приобретает скорость $u$ в
направлении оси $y$, и продолжает двигаться с этой постоянной скоростью.

\begin{figure}[h]
  \centering
  \subfloat[Вид сбоку.]{\label{fig:plate1}\begin{tikzpicture}
    \draw[->] (0,-1) -- (0,5) node[right] {$y$};
    \draw[->] (-1,2) -- (6,2) node[below] {$x$};
    \draw[marrow,red,line width=0.1cm] (0,-0.5) -- (0,4.5) node[left]
    {$\vec{J}$};
    \foreach \x in {1,2,3,4} {\draw[thick,blue,marrow] (\x,4.5) --
      (\x,-0.5);};
    \draw[line width=0.1cm] (5,5) -- (5,-1);
    \draw[very thick,->] (5,3.5) -- (6,3.5) node[above] {$v$};
    \draw[blue] (3.5,4.7) node {$\vec{E}$};
    \draw[green!40!black] (3.4,0.1) -- ++(0.2,-0.2);
    \draw[green!40!black] (3.4,-0.1) -- ++(0.2,0.2);
    \draw[green!40!black] (3.5,0) circle (0.2) node[below=0.1cm]
    {$\vec{B}$};
    \draw[green!40!black] (-0.6,0) circle (0.2) node[below=0.1cm]
    {$\vec{B}$};
    \draw[fill=green!40!black] (-0.6,0) circle (0.05);
    \draw[thick,dashed] (3.1,-0.9) rectangle (6,1) node[below left] {$\Gamma_1$};
  \end{tikzpicture}}
  \hspace{1cm}
  \subfloat[Вид сверху.]{\label{fig:plate2}
    \begin{tikzpicture}
      \draw[->] (0,-1) -- (0,5);
      \draw[->] (-1,2) -- (6,2) node[below] {$x$};
      \draw[red,line width=0.1cm] (0,-0.5) -- (0,4.5)
      node[midway,above left]
      {$\vec{J}$};
      \draw[very thick,red] (0,2) circle (0.25cm);
      \draw[red,fill=red] (0,2) circle (0.1cm);
      \draw[green!40!black,thick,marrow] (-0.75,4.5) -- (-0.75,-0.5)
      node[below] {$\vec{B}$};
      \foreach \x in {1,2,3,4} {\draw[thick,green!40!black,marrow] (\x,-0.5) --
        (\x,4.5);};
      \draw[green!40!black] (2,-0.5) node[below] {$\vec{B}$};
      \draw[blue] (3.4,0.1) -- ++(0.2,-0.2);
      \draw[blue] (3.4,-0.1) -- ++(0.2,0.2);
      \draw[blue] (3.5,0) circle (0.2) node[below=0.1cm] {$\vec{E}$};
      \draw[line width=0.1cm] (5,5) -- (5,-1);
      \draw[very thick,->] (5,3.5) -- (6,3.5) node[above] {$v$};
      \draw[thick,dashed] (3.1,-0.9) rectangle (6,1) node[below left] {$\Gamma_2$};
    \end{tikzpicture}
}
  \label{fig:moving_plate}
\end{figure}

Итого мы получаем ток $\vec{J}$ в направлении оси $y$. Коль скоро у
нас есть ток, то будет и магнитное поле, устроенное также, как и поле
от провода --- при $x>0$ оно направлено от нас, при $x<0$ --- на нас. 

Ясно, что если магнитное поле скачком меняется от нуля до конечной
величины, то появиляются очень большие электрические эффекты. Так что
появляется меняющееся магнитное поле и меняющееся электрическое. Таким
образом, возникает производная $\pt \vec{E} / \pt t$, которая вместе с
током $\vec{J}$ будет вносить вклад в магнитное поле (по уравнению
Максвелла). Выходит, что уравнения очень сильно зацеплены друг с
другом и сильно зависят от времени. 

Попробуем, однако, описать количественно, что происходит. Посмотрим на
эту систему сверху. Ток будет направлен на нас, магнитное поле лежит
в плоскости рисунка. Если смотреть сбоку, то лист будет двигаться
вверх, а поле будет смотреть на нас или от нас. 

Предположим такое распределение полей. Пусть с момента начала движения
нашей плоскости прошло время $t$. Тогда утверждается, что поля
$\vec{E}, \vec{B}$ будут существовать в пространстве только до
$x=vt$, где $v$ --- некоторая константа. Проверим, как это согласуется с
уравнениеями Максвелла. 

Проведём прямоугольный контур $\Gamma_1$ (см. рисунок \ref{fig:plate1}). Он
охватывает часть пространства, где есть поля (слева от прямой $x=vt$),
и часть пространства, где нет полей. Если фронт движется со скоростью
$v$, то поток магнитного поля через $\Gamma_1$ будет меняться тоже со
скоростью $v$. Если ширина прямоугольника равна $L$, то наводящаяся по
контуру ЭДС равна, по закону Фарадея,

\begin{equation}
  \label{eq:moving_plate_1}
  E = v B.
\end{equation}

То есть, если отношение $E$ к $B$ равно $v$, то наши поля будут
удовлетворять закону Фарадея (т.е. одному из уравнений Максвелла). 

Однако, у нас есть ещё одно уравнение: оно связывает ротор магнитного
поля с изменением электрического и током. Чтобы применить его,
посмотрим на нашу систему сверху. Опять нарисуем прямоугольный контур
$\Gamma_2$ (см. рисунок \ref{fig:plate2}), который пересекает волновой
фронт. Токов через этот контур не проходит, поэтому циркуляция $B$
равна скорости изменения потока $E$. Аналогично предыдущим
соображениям, находим, что

\begin{equation}
  \label{eq:moving_plate_2}
  B = \frac{v}{c^2} E.
\end{equation}

Сранивая уравнения \eqref{eq:moving_plate_1} и
\eqref{eq:moving_plate_2} видим, что фронт распространяется со
скоростью $v=c$. Как мы помним, $c$ было просто некоторой константой,
введённой для удобства. Теперь оказывается, что эта константа
совпадает со скоростью распространения электромагнитной
волны. Разумеется, это не случайное совпадение --- фактически мы
подразумевали это в разделе \ref{sec:maxwell_eq_4}, когда встраивали
электромагнитные волны в нашу теорию. Теперь мы убедились, как это
работает в конкретном примере. 

Заметим попутно, что эта скорость может быть вычислена ещё и таким
образом: $c= 1/ \sqrt{\eps_0 \mu_0}$. Константы $\eps_0, \mu_{0}$
могут быть измерены экспериментально --- из взаимодействия зарядов и
токов, например. Измерения дают, что $c$ с огномной точностью равна
скорости света, и это, разумеется, не случайное совпадение. Отсюда,
кстати, ещё следует, что эта скорость зависит только от свойств среды.

Итак, мы выяснили, что распространение фронта электромагнитной волны
происходит с конечной скоростью; эта скорость является скоростью
света; кроме того, эта скорость зависит только лишь от материальных
свойств среды.

Можно ещё задаться таким вопросом: что произойдёт, если спустя
некоторое время $T$ после начала движения, мы остановим нашу плоскость
с током? Ясно, что мы можем смоделировать эту ситуацию, запуская
противоположно заряженную плоскость в противиположном направлении ---
в этом случае суммарный ток будет равен нулю. 

Эта противоположно заряженная плоскость сгенерирует также
электрическое и магнитное поля, только противоположно
направленные. Таким образом, через время $T$ в окрестности плоскости
никаких полей не будет (они скомпенсируются) и эта картина будет со
скоростью $c$ распространяться вдоль оси $x$. Таким образом, мы
получим электромагнитный импульс ширины $cT$, который будет двигаться
в пространстве со скоростью $c$. 

Каким образом этот импульс сам себя поддерживает? Ведь он существует в
отрыве от источников тока и зарядов. Ответ такой: за счёт сочетания
зацепленных уравнений Максвелла. Предположим, что магнитное поле
исчезло бы; тогда, по закону Фарадея, изменилось бы электрическое поле
(так как изменился магнитный поток). Но в этом случае изменился бы
электрический поток, и по другому уравнению Максвелла появилось бы
магнитное поле, и т.д. Таким образом, электромагнитная волна
представляет собой самоподдерживающуюся систему, которая может
существовать в отрыве от зарядов и токов.

\com{Комментарий про перпендикулярность E и B; возможно, пара слов о поляризации}

\end{document}