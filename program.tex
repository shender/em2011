\documentclass[11pt]{article}
% math symbols
\usepackage{amssymb,amsmath}
% for different compilers
\usepackage{ifpdf}
% geometry of page
\usepackage[margin=2.1cm]{geometry}
% float pictures
\usepackage{wrapfig}

% if pdflatex, then
\ifpdf
 \usepackage[english,russian]{babel}
 \usepackage[utf8]{inputenc}
 \usepackage[unicode]{hyperref}
 \usepackage[pdftex]{graphicx}
 \usepackage{cmlgc}
% if xelatex, then
\else
% math fonts
 \usepackage{fouriernc}
% xelatex specific packages
 \usepackage[xetex]{hyperref}
 \usepackage{xunicode}	% some extra unicode support
 \usepackage{xltxtra}	% \XeLaTeX macro
 \defaultfontfeatures{Mapping=tex-text}
 \usepackage{polyglossia}	% instead of babel in xelatex
 \setdefaultlanguage{russian}
% fonts
 \setromanfont{Charis SIL}
 \setsansfont{OfficinaSansC} 
 \setmonofont{Consolas}
\fi

% several pictures in one figure
\usepackage{subfig}
% calc in TeX expressions
\usepackage{calc}
% nice pictures and plots
\usepackage{pgfplots,tikz,circuitikz}
% different libraries for pictures
\usetikzlibrary{%
  arrows,%
  calc,%
  patterns,%
  decorations.pathreplacing,%
  decorations.pathmorphing,%
  decorations.markings%
}
\tikzset{>=latex}

% colors of the hyperlinks
\hypersetup{colorlinks,%
  citecolor=blue,%
  urlcolor=blue,%
  linkcolor=red
}

\tolerance=1000
\emergencystretch=0.74cm

\newcommand{\nn}{\nonumber}
\newcommand{\pt}{\partial}
\newcommand{\eps}{\epsilon}
\newcommand{\vareps}{\varepsilon}
\newcommand{\const}{\mathrm{const}}
\newcommand{\com}[1]{{\Large{\texttt{{\color{red}(#1)}}}}}

\pagestyle{empty}
\begin{document}
\begin{center}
  \LARGE{\textbf{Электродинамика и теория относительности}}\\[1cm]
  \Large{Дмитрий Соколов, Игорь Шендерович}\\[1cm]
\end{center}

\begin{abstract}
  В рамках основных занятий по электродинамике мы будем изучать
  основные свойства электрических и магнитных полей, и в особенности
  их зависимость от времени (то есть, \textit{динамику} этих
  полей). Уравнения, описывающие эту динамику, известны как
  \textit{уравнения Максвелла} — они были получены Дж. Максвеллом во
  второй половине XIX века. Кроме того, мы увидим, что логическое
  продолжение этой деятельности приводит к появлению
  \textit{специальной теории относительности}, сформулированной
  Эйнштейном в 1905 году.
\end{abstract}

\begin{center}
  \Large{\textbf{План занятий}}
\end{center}

\begin{enumerate}
\item Напоминание об электростатике. Основные векторные операции.
\item Метод изображений. Гидромеханическая аналогия.
\item Движущиеся электрические заряды.
\item Магнитное поле как следствие динамики. Неразличимость
  электрического и магнитного поля. 
\item Первое динамическое уравнение Максвелла. Задача о разрядке
  конденсатора. 
\item Электромагнитные волны.
\item Закон индукции Фарадея: второе динамическое уравнение
  Максвелла.
\item Принцип относительности: сохранение формы уравнений Максвелла. 
\item Преобразования Лоренца.
\item Физические следствия преобразований Лоренца. 
\end{enumerate}



\end{document}
